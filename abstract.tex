Sentiment analysis aims to identify the attitudes or emotions behind texts. For many sentiment analysis approaches, sentiment information of terms or phrases plays an important role. However, in Chinese sentiment analysis, the coverage of such information is still limited. To increase the coverage, some methods has been developed to predict sentiments for nodes in Chinese ConceptNet due to its large size and high semantic level nodes. In ConceptNet, the notion of a node is extended from purely lexical terms to include higher-order compound concepts, e.g., 'eat lunch', 'satisfy hunger', so these nodes are called concepts. 

Current approaches aim to assign one sentiment to each concept, but in fact a concept may have different sentiments on different contexts, such as 'scream' and 'sudden' in Chinese. In this paper, our first goal is to extract the hidden contextual information in Chinese ConceptNet and use it to estimate sentiments in different situations for each concept. To achieve this goal, we propose a topic-aware sentiment propagation approach. We apply Latent Dirichlet Allocation to divide Chinese ConceptNet into different topic layers and use sentiment propagation on each topic layer to predict topic-aware sentiments for Chinese concepts. Our another goal is to use the generated topic-aware sentiments of concepts to improve the polarity classification for texts. We combine other co-occurring concepts to identify topics and select sentiments for concepts in texts. Then, experiments conducted on dialogue dataset and microblog posts show the improvement of topic-aware prediction for concepts and texts. 