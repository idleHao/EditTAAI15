\subsection{Building Sentiment Dictionaries}
\subsubsection{Manually}
One way to build a sentiment dictionary is human annotation. The Affective Norms for English Words (ANEW) ~\cite{Bradley:ANEW99} is compiled manually and provides a set of normative ratings for 1034 English words. Bradley and Lang asked a group of Introductory Psychology class students to rate words in the ANEW through a normative rating procedure. They describe emotion using PAD emotional state model, and 

They use a picture-oriented instrument called the Self-Assessment Manikin (SAM) ~\cite{Lang:behavioral80} as the affective rating system to collect the pleasure, arousal, and dominance information for words. Although the sentiment values labelled by humans have high quality, the cost of collecting these labels is high. As the result, the coverage of words is limited.

\subsubsection{Sentiment Propagation}
Random walk is commonly used to spread sentiment values on a semantic graph~\cite{Wu:relSelect14, Hassan:ACL10, Xu:COLING10, Cambria:AAAI10}. Sentiment values are propagated through sentiment related edges, such as synonym and antonym relation. The equation of random walk with restart is as follows:
\begin{equation}
\label{eq:rndWalk}
\boldsymbol{s}_{t+1} = (1-\alpha)\boldsymbol{W}\boldsymbol{s}_t + \alpha\boldsymbol{s}_0
\end{equation}
where $\boldsymbol{s}_t$ is the values of each node when $t$-th iteration. $\boldsymbol{W}$ is a similarity matrix, which can be acquired from sources like Ontology, ConceptNet, corpus, etc. Similarity matrix of a standard random walk is an out-link normalized matrix. $\alpha$ is the restarting weight. Random walk is an iterative process, and after $n$ iteration, each node spreads its value to the neighbors that are $n$ links distant from it.

Besides terms, concepts like 'eat lunch', 'satisfy hunger' are associated with sentiments. Because the coverage of nodes in ConceptNet contains not only lexical terms but also such higher-order compound concepts, there are approaches proposed to predict sentiment values for nodes in ConceptNet~\cite{Liu:IUI03, Cambria:AAAI10, Wu:TAAI11, Tsai:IEEE13, Wu:relSelect14}. In ~\cite{Tsai:IEEE13, Wu:relSelect14}, random walk is applied to propagate sentiment values through relations of ConceptNet. They found that in ConceptNet, performing in-link normalized on similarity matrix is better than performing out-link normalized because out-link normalization will underestimate the influence of concepts with more neighbors. In in-link normalization, each concept's new sentiment value in the ($t+1$)-th iteration is the average of all its neighbors in the $t$-th iteration. However, in ConceptNet, a concept's neighbors may come from different situations, but this is not considered.

