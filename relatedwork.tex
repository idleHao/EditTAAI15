\subsection{Building Sentiment Dictionaries}
\subsubsection{Manually}
One way to build a sentiment dictionary is by human labelling. The Affective Norms for English Words (ANEW) ~\cite{Bradley:ANEW99} is a famous sentiment dictionary compiled manually and provides a set of normative ratings for 1034 English words. Bradley and Lang asked a group of Introductory Psychology class students to rate words in the ANEW through a normative rating procedure. They describe emotion using PAD emotional state model, and use a picture-oriented instrument called the Self-Assessment Manikin (SAM) ~\cite{Lang:behavioral80} as the affective rating system to collect the pleasure, arousal, and dominance values for English words. However, the amount of words labelled is limited. Next, we introduce approaches using sentiment propagation to scale up sentiment dictionaries.

\subsubsection{Sentiment Propagation}
Random walk is commonly used to spread values on a network. Some approaches~\cite{Wu:relSelect14, Hassan:ACL10, Xu:COLING10, Cambria:AAAI10} starts from a set of words and their sentiments, and uses random walk to expand the set. The equation of random walk with restart is as follows:
\begin{equation}
\label{eq:rndWalk}
\boldsymbol{s}_{t+1} = (1-\alpha)\boldsymbol{W}\boldsymbol{s}_t + \alpha\boldsymbol{s}_0
\end{equation}
where $\boldsymbol{s}_t$ is the values of each node when $t$-th iteration. $\boldsymbol{W}$ is a similarity matrix, which can be acquired from sentiment related edges built from semantic networks or corpus. Similarity matrix of a standard random walk is an out-link normalized matrix. $\alpha$ is the restarting weight. Random walk is an iterative process, and after $n$ iteration, each node spreads its value to the neighbors that are $n$ links distant from it.

Besides terms, concepts like ``eat lunch", ``satisfy hunger" are associated with sentiments. To obtain the sentiments of concepts, approaches are proposed to predict sentiment values for nodes in ConceptNet~\cite{Liu:IUI03, Cambria:AAAI10, Wu:TAAI11, Tsai:IEEE13, Wu:relSelect14}. The reason is that the coverage of nodes in ConceptNet contains not only lexical terms but also such higher-order compound concepts. In ~\cite{Tsai:IEEE13, Wu:relSelect14}, random walk is applied to propagate sentiment values through relations of ConceptNet. They found that in ConceptNet, performing in-link normalized on similarity matrix is better than performing out-link normalized because out-link normalization will underestimate the influence of concepts with more neighbors. In in-link normalization, each concept's new sentiment value in the ($t+1$)-th iteration is the average of all its neighbors in the $t$-th iteration. However, in ConceptNet, a concept's neighbors may come from different contexts, but this is not handled in previous sentiment propagation.

In addition, in the previous sentiment propagation, each concept will be associated with one sentiment value in the end. However, a concept's sentiment value should be different in different contexts or domains. For example, ``scream" and ``sudden" are both possible to be positive and negative. 

\subsubsection{Context-Aware Sentiment Dictionaries}
Common approaches aiming to build context-aware sentiment dictionaries rely on corpus and heuristic rules of sentences~\cite{Xu:PACLIC10, Xu:COLING10, Lu:WWW11}. They use the structures of sentences to construct relationship between words, such as ``and" and ``but" connecting two adjectives. Another approach~\cite{Boia:AAAI14} is using a game to acquire contexts and sentiments of ambiguous phrases from humans. 

Rao et al.~\cite{Rao:WWW14} use topic model to analyze co-occurrence information of the given news corpus and infer the topic distribution of each article. Combined with emotion labels, they can generate a sentiment dictionary which associates a sentiment value with each topic. 

Besides corpus, ConceptNet also has abundant contextual information. Its semantic relations between concepts, which are not limited to synonym and antonym, can be used to propagate sentiments. For example, ``get F score Causes sad" tells us that the sentiment of concept ``get F score" is related to concept ``sad" because they are connected by the relation ``Causes". As a result, our approach relies on the contextual information and structure of ConceptNet to predict topic-aware sentiments for Chinese concepts.